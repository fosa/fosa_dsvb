% coding:utf-8

%----------------------------------------
%FOSADSVB, a LaTeX-Code for a summary of digital signal processing
%Copyright (C) 2015, Mario Felder & Michi Fallegger

%This program is free software; you can redistribute it and/or
%modify it under the terms of the GNU General Public License
%as published by the Free Software Foundation; either version 2
%of the License, or (at your option) any later version.

%This program is distributed in the hope that it will be useful,
%but WITHOUT ANY WARRANTY; without even the implied warranty of
%MERCHANTABILITY or FITNESS FOR A PARTICULAR PURPOSE.  See the
%GNU General Public License for more details.
%----------------------------------------

\chapter{Matlab}

\begin{tabular}{lp{5cm}p{2.2cm}} 
\multicolumn{3}{l}{\textbf{Matrizengenerierung}} \\\toprule
Befehl & Beschreibung & Syntax \\ \midrule
\verb|zeros| & Matrix mit Nullen & \verb|zeros(N)|, \verb|zeros(M,N)| \\ 
\verb|ones| & Matrix mit EInsen & \verb|ones(N)|, \verb|ones(M,N)| \\ 
\verb|rand| & Matrix mit gleichverteilten Zufallszahlen & \verb|rand(N)|, \verb|rand(M,N)| \\ 
\verb|randn| & Matrix mit normalverteilten Zufallszahlen & \verb|randn(N)|, \verb|randn(M,N)| \\ 
\verb|linspace| & Matrix mit linear steigenden Werten & \verb|linspace(N)|, \verb|linspace(M,N)| \\ 
\verb|logspace| & Matrix mit logarithmisch steigenden Werten & \verb|logspace(N)|, \verb|logspace(M,N)| \\ \bottomrule
\end{tabular} 

\begin{tabular}{lp{5cm}p{2.2cm}} 
\multicolumn{3}{l}{\textbf{Elementar Funktionen}} \\\toprule
Befehl & Beschreibung & Syntax \\ \midrule
\verb|log| & natürlicher Logarithmus & \\ 
\verb|log10/2| & Logarithmus zur Basis 10 bzw. 2 & \\ 
\verb|round| & Rundung zur nächsten ganzen Zahl &  \\ 
\verb|ceil| & Rundung Richtung $\infty$ & \verb|ceil(n.x)| = $n+1$ \\ 
\verb|floor| & Rundung Richtung $-\infty$ & \verb|floor(n.x)| = $n$ \\ 
\verb|mean| & Arithmetische Mittel & \\ \bottomrule
\end{tabular}

\begin{tabular}{lp{5cm}p{2.2cm}}
\multicolumn{3}{l}{\textbf{Komplexe Zahlen}} \\\toprule
Befehl & Beschreibung & Syntax \\ \midrule
\verb|abs| & Betrag & \\ 
\verb|angle| & Phase & \\ 
\verb|real| & Realanteil &  \\ 
\verb|imag| & Imaginäranteil &  \\ 
\verb|conj| & conjungiert complex & \\ 
\verb|polar| & Polarkoordinaten-Darstellung & \\ \bottomrule
\end{tabular} 

\begin{tabular}{lp{5cm}p{2.2cm}}
\multicolumn{3}{l}{\textbf{Graphik}} \\\toprule
Befehl & Beschreibung & Syntax \\ \midrule
\verb|figure| & Grafik-Fenster & \\ 
\verb|plot| & 2-D Grafik & \verb|plot(X1,Y1,S1,X2,Y2,S2,...)|\\ 
\verb|subplot| & Unterteilung des Grafik-Fensters & \verb|subplot(z,s,p)| \\ 
\verb|stem| & Darstellung diskreter Werte &  \\ 
\verb|hold| & Halten der Grafik zum Übereinanderzeichnen & \\ \bottomrule
\end{tabular} 

\begin{tabular}{lp{5cm}p{2.2cm}}
\multicolumn{3}{l}{\textbf{Transformation und Filterung}} \\\toprule
Befehl & Beschreibung & Syntax \\ \midrule
\verb|dftmtx| & Transformationsmatrix für DFT & \verb|dftmtx(N)| \\ 
\verb|fft/fft2| & schnelle DFT (FFT), 1-/2-dimensional & \\ 
\verb|ifft/ifft2| & inverse FFT, 1-/2-dimensional & \\ 
\verb|fftshift| & Verschiebung des Gleichanteils in die Mitte des Spektrums &  \\ 
\verb|spectrum| & Leistungsdichte-Spektrum & \\ 
\verb|filter| & Filterung & \verb|filter(B,A,X)| ($H(z) = B(z)/A(z)$) \\
\verb|conv| & Faltung & \\
\verb|fftfilt| & schnelle Faltung & \\
\verb|freqz| & Frequenzgang & \verb|freqz(B,A,N)| \\ 
\verb|impz| & Impulsantwort & \verb|impz(B,A,N)| \\ 
\verb|zplane| & Pol/Nullstellenverteilung & \verb|zplane(B,A)| \\\bottomrule
\end{tabular} 

\begin{tabular}{lp{5cm}p{2.2cm}}
\multicolumn{3}{l}{\textbf{Filterentwurf}} \\\toprule
Befehl & Beschreibung & Syntax \\ \midrule
\verb|fir1| & Hoch-, Tief-, Bandpass, Bandsperre & \verb|fir1(N,Wn,'low')|
 \verb|fir1(N,Wn,'high')| 
 \verb|Wn=[W1,W2]| \verb|fir1(N,Wn,'bandpass')| 
 \verb|fir1(N,Wn,'stop')|\\ 
\verb|fir2| & allgemeiner Frequenzverlauf & \\ 
\verb|butter| & Butterworth IIR & \verb|[B,A]=butter(N,Wn,'low')| \\ 
\verb|cheby1| & Chebyshev I IIRr & \\
\verb|cheby2| & Chebyshev II IIRr & \\ 
\verb|ellip| & Elliptisch Cauer & \\\bottomrule
\end{tabular} 

\begin{tabular}{lp{5cm}p{2.2cm}}
\multicolumn{3}{l}{\textbf{Fensterung}} \\\toprule
Befehl & Beschreibung & Syntax \\ \midrule
\verb|rectwin| & Rechteckfenster & \verb|rectwin(N)| \\ 
\verb|bartlett| & Dreieck-Fenster & \verb|bartlett(N)|\\ 
\verb|hamming| & Hamming-Fenster & \verb|hamming(N)| \\ 
\verb|kaiser| & Kaiser-Fenster & \verb|kaiser(N)| \\ \bottomrule
\end{tabular} 